         \documentclass{article}
\usepackage[T2A]{fontenc}
\usepackage[utf8]{inputenc}
\usepackage[russian]{babel}
\usepackage[normalem]{ulem}
\usepackage{amsmath}
\usepackage{ amssymb }
\usepackage{ wasysym }
\usepackage{graphicx}
\graphicspath{ {images/} }
\usepackage{array,epsfig}
\usepackage{amsmath}
\usepackage{ dsfont }
\usepackage{ textcomp }
\usepackage{amsfonts}
\usepackage{ gensymb }
\usepackage{amssymb}
\usepackage{amsxtra}
\usepackage{amsthm}
\usepackage{mathrsfs}
\usepackage{color}
\usepackage{enumitem}
\usepackage{graphicx}
\usepackage{amsmath}
\newcommand\mymathop[1]{\mathop{\operatorname{#1}}}
\DeclareGraphicsExtensions{.pdf,.png,.jpg}
\newtheorem{solution}{Решение}
\newtheoremstyle{problemstyle}  % <name>
        {3pt}                                               % <space above>
        {25pt}                                               % <space below>
        {\normalfont}                               % <body font>
        {}                                                  % <indent amount}
        {}                 % <theorem head font>
        {\normalfont\bfseries.}         % <punctuation after theorem head>
        {.5em}                                          % <space after theorem head>
        {}                                                  % <theorem head spec (can be left empty, meaning `normal')>
\theoremstyle{problemstyle}
\newtheorem{problem}{\bfseriesЗадача}
\newtheoremstyle{ans}  % <name>
        {3pt}                                               % <space above>
        {3pt}                                               % <space below>
        {\normalfont}                               % <body font>
        {}                                                  % <indent amount}
        {}                 % <theorem head font>
        {\normalfont\bfseries}         % <punctuation after theorem head>
        {.5em}                                          % <space after theorem head>
        {}                                                  % <theorem head spec (can be left empty, meaning `normal')>
\theoremstyle{ans}
\newtheorem{cor}{Corollary}
\newtheorem*{ans}{Ответ}
\newtheorem{lem}{Lemma}
\newtheorem*{joke}{Joke}
\newtheorem{ex}{Example}
\newtheorem*{soln}{\bfseries\textit{Ответ}:}
\newtheorem{prop}{Proposition}
\newtheorem*{zad}{\zadname}
\newtheorem{definition}{\bfseries\textit{Опредление}}
\newtheorem{theorem}{\bfseries\textit{Теорема}}
\newtheorem{example}{\textit{Пример}}
\newtheorem{comment}{\bfseries{Замечание}}
\newtheorem{train}{\textit{Упражнение}}
\newtheorem{statement}{\textbf{Утверждение}}
\newtheorem*{statement*}{ \textbf{Утверждение}}
\newtheorem{conseq}{\textbf{Следствие}}
\newtheorem*{conseq*}{\textbf{Следствие}}

\newcommand{\N}{\mathbb{N}}
\newcommand{\Expect}{\mathsf{E}}
\newcommand{\R}{\mathbb{R}}
\newcommand{\U}{\mathbb{U}}



\usepackage{xcolor}
\usepackage{hyperref}

\usepackage[unicode, pdftex]{hyperref}

\usepackage{geometry}
 \geometry{
 a4paper,
 total={170mm,257mm},
 left=20mm,
 top=20mm,
 }
%---------------------------------------------------------
\usepackage{amsfonts}
%Hyphenation rules
%---------------------------------------------------------
\usepackage{hyphenat}
\hyphenation{ма-те-ма-ти-ка вос-ста-нав-ли-вать}
 \usepackage{titlesec}
\titlelabel{\thetitle.\quad}


\title{Лекция 1 по введению в IOT}
\author{Мячин Данил БПМИ187}
\date

\begin{document}
\begin{center}
Игорь Рубенович Агамирзян\\
Введение в IOT/CPS
\end{center}
\tableofcontents
\newpage
\section{Лекция 1. Вводная}
Ничего особо не было, просто поговорили, познакомиилсь.
\newpage
\section{Лекция 2. О том, как автоматизация пришла в наш мир}
Субтрактивные методы обработки - убирают материал и делают изделие.\\
Аддитивные методы обработки - наращивают материал и делают изделие.

\subsection{Киберфизические системы}
\begin{itemize}
    \item Либой станок или робот ялвяется киберфизической системой
    \item Киберфизическая система состоит из сенсоров (датчиков), контроллеров (вычислительных блоков) и актуторов (исполнительных элементов)
    \item Примеры киберфизических систем - станки? роботы 
\end{itemize}

\subsection{Прицнип цифрового кодирование}

Что может/умеет делать выислительный блок (компьютер, микроконтроллер и т.д)?

\begin{itemize}
    \item Считывать кодированный поток цифровых данных от сенсоров
    \item Обрабатывать полученные данные в соответствии с програмной логикой и принимать реешния, базируясь на полученных данных
    \item Передавать кодированный поток цифровых данных на исполнение актуаторам
\end{itemize}

Это совершенно универсальный принцип - другого не бывает.

Цифровый данные всегда кодируются значениями да/нет

\begin{itemize}
    \item 0/1
    \item true/false
    \item есть отверстие/нет отверстия (перфокарта)
    \item есть нажатие клавиши/нет нажатия клавиши
    \item пиксель освещён/пиксель погашен
\end{itemize}


\subsection{NC и CNC - компьютер как инструмент управления}

% --- тут опечатка на слайде
\begin{itemize}
    \item Принцип управления первых механических и электронных станков, использовавших Жаккардов принцип - управления по программе с помощью перфокарт или перфоленты - получил название Numerical Control (NC)
    \item Использование компьютера для управления, принятый в киберфизических системах принято называть Computer Numerical Control. (CNC)
    \item Разница - в наличии сенсоров, позволяющих организовать интеллектуальную обратную связь
    \item Прогресс в микроэлектронике (уменьшение размеров и энергопотребления при одновременном росте вычислительной мощности контроллеров) позволил реализовать весь современный тенхнологический ландшафт
\end{itemize}


\subsection{Управляющее программное обеспечение}

\begin{itemize}
    \item Вторым необходимым компонентом современных технологий является программное обеспечение
    \item Сегодня не существует высокотехнологичных устройств, не содержащих внутри себя микроконтроллеров и программного обеспечения
    \item Если на устройстве есть дисплей, значит, внутри него есть микроконтроллер и управляющее программное обеспечение
\end{itemize}

Всё современное программное ообеспечение разделилось на несколько классов, для которых используются различные языки и среды программирования

\begin{itemize}
    \item Инфраструктурное ПО - операционные системы, базы данных и т.д.
    \item Корпоративное ПО - это весь набор АСУ (ERP, CRM, и т.д.)
    \item WEB ПО - серверы и HTML 5.0, клиенсткие программы на Java и JS
    \item Встроенное (embedded) ПО - АСУТП, именно это программное обеспечение управляет миром
\end{itemize}

\subsection{Кодирование цифровых элекронных сигналов}
Пикча момент

\begin{itemize}
    \item В электронных устройствах цифровые значения кодируются уровнями напряжений в текущей момент времени
    \item Существует ряд стандартов, в которых, как правило, логический 0 кодируется напряжением 0 V, а логическая единица - значением +5V для TTL-kлогики или +3.3 V для CMOS-логики
    \begin{itemize}
        \item TTL \href{https://www.youtube.com/watch?v=qOtHc_x70hY}{15-ти минутная лекция}
        \item CMOS \href{https://www.youtube.com/watch?v=9eIkCWrPIqY}{15-ти минутная лекция}
    \end{itemize}
\end{itemize}


\subsection{Кодирование логических значений уровнями напряжения}
\begin{itemize}
    \item Последовательности цифровых (двоичных) данных кодируются уровнями напряжений 0-5 V с определённой частотой дискретизации (квант времени лии такт)
    \item Принято называть эти уровни HIGH для наличия сигнала (2-5 V) и LOW для его отсутствия (0-0.8V)
    \item В случае, еесли уровень меняется в течение такта, то сигнал может быть не определён
\end{itemize}

\subsection{Транзистор (упрощённо для цифровой логики)}
\href{https://www.youtube.com/watch?v=X99j9CVvf1w}{Какой-то видос на ютубе на 4 минуты} 
\begin{itemize}
    \item В цифровой TTL-логике тразистор может упрощённо считаться управляемым выключателем (ключом)
    \item Если на базу тразистора подан сигнал HIGH, выключатель включён
    \item Если на базу тразистора подан сигнал LOW, выключатель выключен
\end{itemize}

\subsection{Основные логические элементы}
\begin{itemize}
    \item Сущестует три основных логических элемента цифровой логики, соответствующие операциям булевой алгебры.
    \item С их помощью можно реализовать любую логическую схему
    \item Для удобства проектирования реализации принято выделять ещё некоторые элементы.
\end{itemize}

\end{document}
